\documentclass[10pt,landscape]{article}
\usepackage{multicol}
\usepackage{calc}
\usepackage{ifthen}
\usepackage[landscape]{geometry}
\usepackage{amsmath,amsthm,amsfonts,amssymb}
\usepackage{color,graphicx,overpic}
\usepackage{hyperref}


\pdfinfo{
  /Title (example.pdf)
  /Creator (TeX)
  /Producer (pdfTeX 1.40.0)
  /Author (Seamus)
  /Subject (Example)
  /Keywords (pdflatex, latex,pdftex,tex)}

% This sets page margins to .5 inch if using letter paper, and to 1cm
% if using A4 paper. (This probably isn't strictly necessary.)
% If using another size paper, use default 1cm margins.
\ifthenelse{\lengthtest { \paperwidth = 11in}}
    { \geometry{top=.5in,left=.5in,right=.5in,bottom=.5in} }
    {\ifthenelse{ \lengthtest{ \paperwidth = 297mm}}
        {\geometry{top=1cm,left=1cm,right=1cm,bottom=1cm} }
        {\geometry{top=1cm,left=1cm,right=1cm,bottom=1cm} }
    }

% Turn off header and footer
\pagestyle{empty}

% Redefine section commands to use less space
\makeatletter
\renewcommand{\section}{\@startsection{section}{1}{0mm}%
                                {-1ex plus -.5ex minus -.2ex}%
                                {0.5ex plus .2ex}%x
                                {\normalfont\large\bfseries}}
\renewcommand{\subsection}{\@startsection{subsection}{2}{0mm}%
                                {-1explus -.5ex minus -.2ex}%
                                {0.5ex plus .2ex}%
                                {\normalfont\normalsize\bfseries}}
\renewcommand{\subsubsection}{\@startsection{subsubsection}{3}{0mm}%
                                {-1ex plus -.5ex minus -.2ex}%
                                {1ex plus .2ex}%
                                {\normalfont\small\bfseries}}
\makeatother

% Define BibTeX command
\def\BibTeX{{\rm B\kern-.05em{\sc i\kern-.025em b}\kern-.08em
    T\kern-.1667em\lower.7ex\hbox{E}\kern-.125emX}}

% Don't print section numbers
\setcounter{secnumdepth}{0}


\setlength{\parindent}{0pt}
\setlength{\parskip}{0pt plus 0.5ex}

%My Environments
\newtheorem{example}[section]{Example}
% -----------------------------------------------------------------------

\begin{document}
\raggedright
\footnotesize
\begin{multicols}{3}


% multicol parameters
% These lengths are set only within the two main columns
%\setlength{\columnseprule}{0.25pt}
\setlength{\premulticols}{1pt}
\setlength{\postmulticols}{1pt}
\setlength{\multicolsep}{1pt}
\setlength{\columnsep}{2pt}

$a+b \in \mathbb{N}$\\
$a \times b \in \mathbb{N}$\\
$a + b = b + a$\\
$(a + b) + c = a + (b + c)$\\
$n \times 1 = n for all n \in \mathbb{N}$\\
$if m \times z = n \times z for some z \in \mathbb{N} then m = n$\\
$a \times (b + c) = (a \times b) + (a \times c)$\\
$m$ is a multiple of $n$ if there is a natural number $r$ such that $m = rn$\\
If $a$ and $b$ are multiplies of $n$ the, for all $x, y \in \mathbb{N}, xa + yb$ is $a$ a multiple of $n$\\
For any natural numbers $m$ and $n$, the statement $m < n$ means that there is some $x \in \mathbb{N}$ such that $m + x = n$\\
If $a < b$ and $b < c$ then $a < c$\\
Given any natural numbers $m$ and $n$, exactly one of the three statements $m<n, m=n, n<m$ is true.\\
Suppose that P(n) is a statement with the following properties:\\
(i) $P(1)$ is true; \textit{induction basis}\\
(ii) if $P(k)$ is true (\textit{induction hypothesis}) then $P(k+1)$ (\textit{induction step}) is true for every $k \in \mathbb{N}$.\\ 
Then $P(n)$ is true for all $n \in \mathbb{N}$\\
strong: (ii) assume $P(i) 1 \leq i \leq k$ is true then $P(k+1)$ is true for every $k \in \mathbb{N}$\\
Let $X$ be a subset of $\mathbb{N}$. An element $l \in X$ is a \textbf{least member} of $X$ if $l \leq x$ for all $x \in X$. An element $g \in X$ is a \textbf{greatest member} of $X$ if $g \geq x$ for all $x \in X$. Often $l$ and $g$ are referred to as the \textbf{minimum} and \textbf{maximum} of $X$.\\
Every non-empty subset of $X$ of $\mathbb{N}$ has a least member.\\
Suppose that $X$ and $Y$ are sets. We say that we have a \textbf{function} $f$ \textbf{from} $X$ \textbf{to} $Y$ if for each $x$ in $X$ we can specify a unque element in $Y$, which we denote by $f(x)$.\\
The function $f$ from $X$ to $Y$ is a \textbf{surjection} if every $y$ in $Y$ is a value $f(x)$ for at least one $x$ in $X$. It is an \textbf{injection} if every $y$ in $Y$ is a value $f(x)$ for at most one $x$ in $X$. It is a \textbf{bijection} if it is both a surjection and an injection, that is, if every $y$ in $Y$ is a value $f(x)$ for exactly one $x$ in $X$.\\
For any set $X$ the function $i : X \to X$ defined by $i(x) = x$ for all $x \in X$ is called the \textbf{identity} function on $X$. If $X$ is a subset of $Y$, the function $j : X \to Y$ defined by $j(x) = x$ is called the \textbf{inclusion} function from $X$ to $Y$.\\
If $f : X \to Y$ and $g : Y \to Z$ are injections, then so is the composite $gf : X \to Z$. If $f$ and $g$ are surjections then so is $gf$. if $f$ and $g$ are bijections then so is $gf$.\\
A function $f : X \to Y$ has an \textbf{inverse} function $g : Y \to X$ if, for all $x$ in $X$ and $y$ in $Y$ $(gf)(x) = x, (fg)(y) = y$. In other words $gf$ is the identity function on $X$ and $fg$ is the identity function on $Y$.\\
A function has an inverse if and only if it is a bijection.\\
Let $m$ be a natural number. Then the following statement is true for every natural number $n$: if there is an injection from $\mathbb{N}_n to \mathbb{N}_m$, then $n \leq m$.\\ 
If there is a bijective correspondence between $S$ and $\mathbb{N}_m$ then we say that $S$ has size, or cardinality, $m$ and we write $|S| = m$.\\
If a set $S$ is such that $|S| = s$ and $|S| = t$, then $s = t$.\\
A set $S$ is \textbf{finite} if it is empty of if $|S| = n$ for some $n \in \mathbb{N}$. A set which is not finite is said to be \textbf{inifite}.\\
The set $\mathbb{N}$ is infinite.\\
If the set $S$ is such that there is a bijection $b : \mathbb{N} \to S$, then $S$ is infinite.\\
A \textbf{relation} $R$ on a set $X$ is a set of ordered pairs of members of $X$.\\
reflexive $xRx$, symmetric $xRy$, transitive $xRy$ and $yRz$ hence $xRz$\\
An \textbf{equivalence relation} is a relation that is reflexive, symmetric and transitive.\\
Let $R$ be an equivalence relation on $X$. A non-empty set $C \subseteq X$ is an \textbf{equivalence class} with respect to $R$ if\\
(i) any two members of $C$ are $R$-related; and\\
(ii) $C$ contains every member of $X$ that is $R$-related to any member of $C$.\\
in symbols, $C$ is such that,\\
if $x \in C$ then $y \in C \iff xRy$\\ 
Given an equivalence relation $R$ on $X$, every member of $X$ is one and only one equivalence class (with respect to $R$).\\
$x + 0 = x$ for every $x \in \mathbb{Z}$//
If $x \times z = y \times z$ and $z \neq 0$ then $x = y$\\
For any $x \in \mathbb{Z}$ there is an element $-x$ of $\mathbb{Z}$ such that $x +(-x)=0$.\\
If $x \leq y$ and $0 leq z$, then $x \times z \leq y \times z$.\\
The integer $b$ is a \textbf{lower bound} for a set $X \subseteq \mathbb{Z}$ if $b \leq x$ for all $x \in X$.\\
If a non-empty set $S \subseteq \mathbb{Z}$ has a lower bound, then $S$ has a least member.\\
Given positive integers $a$ and $b$ there exist $q$ and $r$ in $\mathbb{N}_0$ such that $a = bq + r$ and $0 \leq r \leq b$. \\
If $a$ and $b$ are positive integers (or zero) we say that $d$ is the \textbf{greatest common divisor (gcd)} of $a$ and $b$ provided that (i) $d|a$ and $d|b$; (ii) if $c|a$ and $c|b$, then $c \leq d$.\\
Let $a$ and $b$ be positive integers, and let $d = gcd(a,b)$. Then there are integers $m$ and $n$ such that $d = ma + nb$.\\
If $gcd(a,b) = 1$ then we say that $a$ and $b$ are \textbf{coprime}. In this case the Theorem asserts that there are integers $m$ and $n$ such that $ma + nb = 1$.\\
A positive integer $p$ is a \textbf{prime} if $p \geq 2$ and the only positive integers which divide $p$ are 1 and $p$ itself.\\
If $p$ is a prime and $x_1, x_2, \dotsc ,x_n$ are any integers such that $p|x_1 x_2 \dots x_n$ then $p|x_i$ for some $x_i(1 \leq i \leq n)$.\\
(The Fundamental Theorem of Arithmetic) A positive integer $n \geq 2$ has a unique prime factorization, apart from the order of the factors.\\

% You can even have references
\rule{0.3\linewidth}{0.25pt}
\scriptsize
\bibliographystyle{abstract}
\bibliography{refFile}
\end{multicols}
\end{document}
