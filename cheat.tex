\documentclass[10pt,landscape]{article}
\usepackage{multicol}
\usepackage{calc}
\usepackage{ifthen}
\usepackage[landscape]{geometry}
\usepackage{amsmath,amsthm,amsfonts,amssymb}
\usepackage{color,graphicx,overpic}
\usepackage{hyperref}


\pdfinfo{
  /Title (example.pdf)
  /Creator (TeX)
  /Producer (pdfTeX 1.40.0)
  /Author (Seamus)
  /Subject (Example)
  /Keywords (pdflatex, latex,pdftex,tex)}

% This sets page margins to .5 inch if using letter paper, and to 1cm
% if using A4 paper. (This probably isn't strictly necessary.)
% If using another size paper, use default 1cm margins.
\ifthenelse{\lengthtest { \paperwidth = 11in}}
    { \geometry{top=.2in,left=.2in,right=.2in,bottom=.2in} }
    {\ifthenelse{ \lengthtest{ \paperwidth = 297mm}}
        {\geometry{top=0.5cm,left=0.5cm,right=0.5cm,bottom=0.5cm} }
        {\geometry{top=0.5cm,left=0.5cm,right=0.5cm,bottom=0.5cm} }
    }

% Turn off header and footer
\pagestyle{empty}

% Redefine section commands to use less space
\makeatletter
\renewcommand{\section}{\@startsection{section}{1}{0mm}%
                                {-1ex plus -.5ex minus -.2ex}%
                                {0.5ex plus .2ex}%x
                                {\normalfont\large\bfseries}}
\renewcommand{\subsection}{\@startsection{subsection}{2}{0mm}%
                                {-1explus -.5ex minus -.2ex}%
                                {0.5ex plus .2ex}%
                                {\normalfont\normalsize\bfseries}}
\renewcommand{\subsubsection}{\@startsection{subsubsection}{3}{0mm}%
                                {-1ex plus -.5ex minus -.2ex}%
                                {1ex plus .2ex}%
                                {\normalfont\small\bfseries}}
\makeatother

% Define BibTeX command
\def\BibTeX{{\rm B\kern-.05em{\sc i\kern-.025em b}\kern-.08em
    T\kern-.1667em\lower.7ex\hbox{E}\kern-.125emX}}

% Don't print section numbers
\setcounter{secnumdepth}{0}


\setlength{\parindent}{0pt}
\setlength{\parskip}{0pt plus 0.5ex}

%My Environments
\newtheorem{example}[section]{Example}
% -----------------------------------------------------------------------

\begin{document}
\raggedright
\footnotesize
\begin{multicols}{4}


% multicol parameters
% These lengths are set only within the two main columns
%\setlength{\columnseprule}{0.25pt}
\setlength{\premulticols}{1pt}
\setlength{\postmulticols}{1pt}
\setlength{\multicolsep}{1pt}
\setlength{\columnsep}{2pt}

\textbf{NATURAL NUMBERS}\\
$a+b \in \mathbb{N}$\\
$a \times b \in \mathbb{N}$\\
$a + b = b + a$\\
$(a + b) + c = a + (b + c)$\\
$n \times 1 = n for all n \in \mathbb{N}$\\
$if m \times z = n \times z for some z \in \mathbb{N} then m = n$\\
$a \times (b + c) = (a \times b) + (a \times c)$\\
$m$ is a multiple of $n$ if there is a natural number $r$ such that $m = rn$\\
If $a$ and $b$ are multiplies of $n$ the, for all $x, y \in \mathbb{N}, xa + yb$ is $a$ a multiple of $n$\\
For any natural numbers $m$ and $n$, the statement $m < n$ means that there is some $x \in \mathbb{N}$ such that $m + x = n$\\
If $a < b$ and $b < c$ then $a < c$\\
Given any natural numbers $m$ and $n$, exactly one of the three
statements $m<n, m=n, n<m$ is true.\\
Suppose that P(n) is a statement with the following properties:\\
(i) $P(1)$ is true; \textit{induction basis}\\
(ii) if $P(k)$ is true (\textit{induction hypothesis}) then $P(k+1)$ (\textit{induction step}) is true for every $k \in \mathbb{N}$.\\ 
Then $P(n)$ is true for all $n \in \mathbb{N}$\\
strong: (ii) assume $P(i) 1 \leq i \leq k$ is true then $P(k+1)$ is true for every $k \in \mathbb{N}$\\
Let $X$ be a subset of $\mathbb{N}$. An element $l \in X$ is a \textbf{least member} of $X$ if $l \leq x$ for all $x \in X$. An element $g \in X$ is a \textbf{greatest member} of $X$ if $g \geq x$ for all $x \in X$. Often $l$ and $g$ are referred to as the \textbf{minimum} and \textbf{maximum} of $X$.\\
Every non-empty subset of $X$ of $\mathbb{N}$ has a least member.\\
\textbf{FUNCTIONS}\\
Suppose that $X$ and $Y$ are sets. We say that we have a \textbf{function} $f$ \textbf{from} $X$ \textbf{to} $Y$ if for each $x$ in $X$ we can specify a unique element in $Y$, which we denote by $f(x)$.\\
The function $f$ from $X$ to $Y$ is a \textbf{surjection} if every $y$ in $Y$ is a value $f(x)$ for at least one $x$ in $X$. It is an \textbf{injection} if every $y$ in $Y$ is a value $f(x)$ for at most one $x$ in $X$. It is a \textbf{bijection} if it is both a surjection and an injection, that is, if every $y$ in $Y$ is a value $f(x)$ for exactly one $x$ in $X$.\\
For any set $X$ the function $i : X \to X$ defined by $i(x) = x$ for all $x \in X$ is called the \textbf{identity} function on $X$. If $X$ is a subset of $Y$, the function $j : X \to Y$ defined by $j(x) = x$ is called the \textbf{inclusion} function from $X$ to $Y$.\\
If $f : X \to Y$ and $g : Y \to Z$ are injections, then so is the composite $gf : X \to Z$. If $f$ and $g$ are surjections then so is $gf$. if $f$ and $g$ are bijections then so is $gf$.\\
A function $f : X \to Y$ has an \textbf{inverse} function $g : Y \to X$ if, for all $x$ in $X$ and $y$ in $Y$ $(gf)(x) = x, (fg)(y) = y$. In other words $gf$ is the identity function on $X$ and $fg$ is the identity function on $Y$.\\
A function has an inverse if and only if it is a bijection.\\
\textbf{HOW TO COUNT}\\
Let $m$ be a natural number. Then the following statement is true for every natural number $n$: if there is an injection from $\mathbb{N}_n to \mathbb{N}_m$, then $n \leq m$.\\ 
If there is a bijective correspondence between $S$ and $\mathbb{N}_m$ then we say that $S$ has size, or cardinality, $m$ and we write $|S| = m$.\\
If a set $S$ is such that $|S| = s$ and $|S| = t$, then $s = t$.\\
A set $S$ is \textbf{finite} if it is empty of if $|S| = n$ for some $n \in \mathbb{N}$. A set which is not finite is said to be \textbf{inifite}.\\
The set $\mathbb{N}$ is infinite.\\
If the set $S$ is such that there is a bijection $b : \mathbb{N} \to S$, then $S$ is infinite.\\
\textbf{INTEGERS}\\
A \textbf{relation} $R$ on a set $X$ is a set of ordered pairs of members of $X$.\\
reflexive $xRx$, symmetric $xRy$, transitive $xRy$ and $yRz$ hence $xRz$\\
An \textbf{equivalence relation} is a relation that is reflexive, symmetric and transitive.\\
Let $R$ be an equivalence relation on $X$. A non-empty set $C \subseteq X$ is an \textbf{equivalence class} with respect to $R$ if\\
(i) any two members of $C$ are $R$-related; and\\
(ii) $C$ contains every member of $X$ that is $R$-related to any member of $C$.\\
in symbols, $C$ is such that,\\
if $x \in C$ then $y \in C \iff xRy$\\ 
Given an equivalence relation $R$ on $X$, every member of $X$ is one and only one equivalence class (with respect to $R$).\\
$x + 0 = x$ for every $x \in \mathbb{Z}$//
If $x \times z = y \times z$ and $z \neq 0$ then $x = y$\\
For any $x \in \mathbb{Z}$ there is an element $-x$ of $\mathbb{Z}$ such that $x +(-x)=0$.\\
If $x \leq y$ and $0 leq z$, then $x \times z \leq y \times z$.\\
The integer $b$ is a \textbf{lower bound} for a set $X \subseteq \mathbb{Z}$ if $b \leq x$ for all $x \in X$.\\
If a non-empty set $S \subseteq \mathbb{Z}$ has a lower bound, then $S$ has a least member.\\
\textbf{DIVISIBILITY AND PRIME NUMBERS}\\
Given positive integers $a$ and $b$ there exist $q$ and $r$ in $\mathbb{N}_0$ such that $a = bq + r$ and $0 \leq r \leq b$. \\
If $a$ and $b$ are positive integers (or zero) we say that $d$ is the \textbf{greatest common divisor (gcd)} of $a$ and $b$ provided that (i) $d|a$ and $d|b$; (ii) if $c|a$ and $c|b$, then $c \leq d$.\\
Let $a$ and $b$ be positive integers, and let $d = gcd(a,b)$. Then there are integers $m$ and $n$ such that $d = ma + nb$.\\
If $gcd(a,b) = 1$ then we say that $a$ and $b$ are \textbf{coprime}. In this case the Theorem asserts that there are integers $m$ and $n$ such that $ma + nb = 1$.\\
A positive integer $p$ is a \textbf{prime} if $p \geq 2$ and the only positive integers which divide $p$ are 1 and $p$ itself.\\
If $p$ is a prime and $x_1, x_2, \dotsc ,x_n$ are any integers such that $p|x_1 x_2 \dots x_n$ then $p|x_i$ for some $x_i(1 \leq i \leq n)$.\\
(The Fundamental Theorem of Arithmetic) A positive integer $n \geq 2$
has a unique prime factorization, apart from the order of the
factors.\\
\textbf{FRACTIONS AND REAL NUMBERS}\\
$\frac{a}{c} \oplus \frac{c}{d} = \frac{ad + bc}{bd}$\\
$\left( \frac{a}{b} \right)^{-1} = \frac{b}{a} (a,b \neq 0)$\\
Between any two rational numbers there is another one.\\
There are no natural numbers $m, n$ such that $m^2 =2n$.\\
Example: 0.5734 meaning $\frac{5}{10} + \frac{7}{100} + \frac{3}{1000}
+ \frac {4}{10000}$.\\
The set of $\mathbb{Q}$ of rational numbers is countable.\\
\textbf{PRINCIPLES OF COUNTING}\\
If $A$ and $B$ are non-empty finite sets, and $A$ and $B$ are disjoint
(that is $A \cap B = \varnothing$, the empty set), then $|A \cup B| =
|A| + |B|$.\\
Let $X$ and $Y$ be finite non-empty sets, and let $S$ be a subset of
$X \times Y$. Then the following results hold.
(i) The size of $S$ is given by $|S| = \displaystyle\sum_{x \in X} r_{x}(S) =
\displaystyle\sum_{y \in Y} c_{y}(S)$ where $r_{x}(S)$ and $c_{y}(S)$
are the row and column totals as described above.
(ii) If $r_{x}(S)$ is a constant $r$, inependent of $x$ and $c_{y}(S)$
is a constant $c$, independent of $y$, then $r|X| = c|Y|$.
(iii) (The multiplication principle) The size of $X \times Y$ is given
by $|X \times Y| = |X| \times |Y|$.\\
\textbf{Eulers's function}: let $\phi(n)$ denote the number of
integers $x$ in the range $1 \leq x \leq n$ such that $x$ and $n$ are
coprime. $\phi(p) = p-1$ ($p$ prime).\\
For any positive integer $n$, $\displaystyle\sum_{d|n} \phi(d)=n$.\\
Let $X$ and $Y$ be non-empty finite sets, and let $F$ denote the set
of functions from $X$ to $Y$. if $|X| = m$ and $|Y| = n$, then $|F| =
n^{m}$. Equivalently, we may say that the number of words of length
$m$ in an alphabet $Y$ of $n$ symbols is $n^{m}$. In general we can
say that a funciton from $\mathbb{N}_{m}$ to $Y$ is a mathematical
model of an \textit{ordered selection with repetition of $m$ things
  from the set $Y$}.\\
The number of ordered selections, without repetition, of $m$ things
from a set $Y$ of size $n$ is the same as the number of injections
from $\mathbb{N}_{m}$ to $Y$, and is given by $n(n-1)(n-2)\dots(n-m+1)$.\\
The following properties hold in the set $S_{n}$ of all permutations
of $\{1,2,\dots,n\}$. (i) If $\pi$ and $\sigma$ are in $S_{n}$, so is
$\pi\sigma$. (ii) For any permutations $\pi, \sigma, \tau$ in $S_{n}$,
$(\pi\sigma)\tau = \pi(\sigma\tau)$.
(iii) The identity function, denoted by id an defined by $id(r) = r$
for all $r$ in $\mathbb{N}_{n}$, is a permutation and for any $\sigma$
in $S_{n}$ we have $id \sigma = \sigma id = \sigma$.
(iv) For every permutation $\pi$ in $S_{n}$ there is an inverse
permutation $\pi^{-1} = \pi^{-1}\pi = id$.\\
\textbf{SUBSETS}\\
\textit{Unordered selection without repetition}: $\binom{n}{r}$ spoken
\textit{n choose r}. \\
If $n$ and $r$ are positive integers satisfying $1 \leq r \leq n$ then
$\binom{n}{r} = \binom{n-1}{r-1} + \binom{n-1}{r}$.\\
If $n$ and $r$ are positive integers satisfying $1 \leq r \leq n$,
then $\binom{n}{r} =
\frac{n(n-1)\dots (n-r+1)}{r!}=\frac{n!}{r!(n-r)!}$.\\
\textit{unordered selections with repetition}, The number of unordered
selections, with repetition, of $r$ objects from a set of $n$ objects
is: $\binom{n+r-1}{r} = \binom{n+r-1}{n-1}$. Since the selections are
unordered, we may arrange matters so that, within each selection, all
the objects of one given kind come first, followed by the objects of
another kind, and so on.\\
Let $n$ be a positive integer. The coefficient of the term
$a^{n-r}b^{r}$ in the expansion of $(a + b)^{n}$ is the binominal
number $\binom{n}{r}$. Explicitly, we have
$(a+b)^{n}=\binom{n}{0}a^{n}+ \binom{n}{1}a^{n-1}b+ \binom{n}{2}a^{n-2}b^{2}+\dots+\binom{n}{n}b^{n}$.\\
If $A_{1},A_{2},\dots,A_{n}$ are finite sets then $|A_{1} \cup A_{2}
\cup \dots A_{n}| =
\alpha_{1}-\alpha_{2}+\alpha_{3}-\dots+(-1)^{n-1}\alpha_{n}$, where
$\alpha_{i}$ is the sum of the cardinalities of the intersections of
the sets taken $i$ at a time $(1 \leq i \leq n)$.\\
Let $n \leq 2$ be an integer whose prime factorization is $n =
p_{1}^{e_{1}}p_{2}^{e_{2}}\dots p_{r}^{e_{r}}$. Then $\phi (n) =
n(1-\frac{1}{p_{1}})(1- \frac{1}{p_{2}})\dots (1- \frac{1}{p_{r}})$.

\textbf{PARTITION, CLASSIFICATION, DISTRIBUTION}\\
Let $S(n,k)$ denote the number of partitions of an n-set $X$ into $k$
parts, where $1 \leq k \leq n$. Then $S(n,1) = 1$, $S(n,n) = 1$,
$S(n,k) = S(n-1, k-1) + kS(n-1,k)$ $(2 \leq k \leq n-1)$.\\
The numbers $S(n,k)$ are sometimes called \textbf{Stirling Numbers}
(of the second kind). As a consequence of Theorem 12.1 they may be
tabulated in much the same way as the binominal numbers are arranged
in Pascal's triangle.\\
If $R$ is an equivalence relation on a set $X$ then the distinct
equivalence classes with respect to $R$ form a partition of $X$.\\
Let $J$ denote the set of surjections from an $n$-set $X$ to a $k$-set
$Y$. Then $|J| = k!S(n,k)$.\\
Given any positive integers $n, n_{1}, \dots ,n_{k}$ satisfying
$n_{1} + n_{2} + \dots + n_{k} = n$, we have
$\binom{n}{n_{1},n_{2},\dots,n_{k}}=\frac{n!}{n_{1}!n_{2}!\dots
  n_{k}!}$.\\
For any positive integers $n$ and $k$ $(x_{1}+ x_{2} + \dots
+ x_{k})^{n} = \sum \binom{n}{n_{1},n_{2}, \dots
  ,n_{k}} x_{1}^{n_{1}} x_{2}^{n_{2}} \dots x_{k}^{n_{k}}$, where the sum
    is taken over all k-tuples of non-negative integers $(n_{1},
    n_{2}, \dots, n_{k})$ such that $n_{1}+n_{2}+ \dots + n_{k} =
    n$.\\
General formula for classification of permutations:
$\frac{n!}{1^{\alpha_{1}}2^{\alpha_{2}}\dots
  n^{\alpha_{n}}\alpha_{1}!\alpha_{2}!\dots \alpha_{n}!}$\\
\textbf{MODULAR ARITHMETIC}\\
Let $x_{1}$ and $x_{2}$ be integers, and $m$ a positive integer. We
say that $x_{1}$ is \textbf{congruent} to $x_{2}$ \textbf{modulo} $m$,
and write $x_{1} \equiv x_{2}$ (mod $m$) whenever $x_{1} - x_{2}$ is
divisible by $m$.\\
Let $m$ be a positive integer and $x_{1}, x_{2}, y_{1}, y_{2}$
integers such that $x_{1} \equiv x_{2}$ (mod $m$), $y_{1} \equiv y_{2}$
(mod $m$). Then (i) $x_{1} + y_{1} \equiv x_{2} + y_{2}$ (mod $m$), (ii)
$x_{1}y_{1} \equiv x_{2}y_{2}$ (mod $m$).\\
The set of \textbf{integers modulo m}, written as $mathbb{Z}_{m}$, is
the set of distinct equivalence classes under the relation of
congurence modulo $m$ in $\mathbb{Z}$.\\
The operations $\oplus$ and $\otimes$ satisfy the following rules,
where $a,b,c$ denote any members of $\mathbb{Z}_{m}$, and $0
= [0]_{m}$, $1 = [1]_{m}$.
\textbf{M1}$a \oplus b$ and $a \otimes b$ are in $\mathbb{Z}_{m}$.
\textbf{M2}$a \oplus b = b \oplus a$, $a \otimes b = b \otimes a$.
\textbf{M3}$(a \oplus b) \oplus c = a \oplus (b \oplus c)$, $(a
\otimes b) \otimes c = a \otimes (b \otimes c)$.
\textbf{M4}$a \oplus 0 = a$, $a \otimes 1 = a$.
\textbf{M5}$a \otimes (b \oplus c) = (a \otimes b) \oplus (a \otimes c)$.
\textbf{M6}For each $a$ in $\mathbb{Z}_{m}$ there is  an unique
element $-a$ in $\mathbb{Z}_{m}$ such that $a \oplus(-a) = 0)$.\\
An element $r$ in $\mathbb{Z}_{m}$ is said to be \textbf{invertible}
if there is some $x$ in $\mathbb{Z}_{m}$ such that $rx = 1$ in
$\mathbb{Z}_{m}$. In the case, $x$ is called the \textbf{inverse} of
$r$, and we write $x = r^{-1}.$.\\
The element $r$ in $\mathbb{Z}_{m}$ is invertible if and only if $r$
and $m$ are coprime in $\mathbb{Z}$. in particular, when $p$ is a
prime every element of $\mathbb{Z}_{p}$ except 0 is invertible.\\
If $y$ is invertible in $\mathbb{Z}_{m}$ then $y^{\phi (m)} = 1$ in
$\mathbb{Z}_{m}$.\\
\textbf{Euler's Theorem}: if gcd($y$,$m$) = 1 then $y^{\phi (m)}
\equiv 1$ (mod $m$).\\
\textbf{Fermat's Theorem}: if $p \nmid y$ then $y^{p-1} \equiv 1$ (mod
$p$).\\
A \textbf{latin square} of order $n$ is an $n \times n$ array in which
each one of $n$ symbols occurs once in each row and once in each
column.\\
Let $p$ be a prime and $t$ a non-zero element of
$\mathbb{Z}_{p}$. Then the rule $L_{t}(i,j) =ti+j$ $(i,j \in
\mathbb{Z}_{p})$ defines a latin square. Furthermore, when $t \nmid u$
the latin square $L_{t}$ and $L_{u}$ are orthogonal.\\
Let $f$ be a function from $\mathbb{N}$ to $\mathbb{N}$. We say that
$f(n)$ is $O(g(n))$ if there is a positive constant $k$ such that
$f(n) \leq kg(n)$ for all $n$ in $\mathbb{N}$ (with possibly a finite
number of exceptions). The symbol $(O(g(n))$ is pronounced `big-oh of
$g(n)$'\\
\textbf{ALGORITHMS AND THEIR EFFICIENCY}\\
\textbf{bubblesort} Compare adjacent terms in the list and switch them
if they are in the wrong order. The operations involved in bubble sort
are comparisons and switches. At the $j$th path, $n-m$ comparisons are
made, adn so the total nmber of comparisons is $(n-1)+(n-2)+ \dots + 2
+ 1 = \frac{1}{2}n(n-1)$, which is $n^{2}$.
\textbf{listsort} The basic idea of insertion sorting is to begin with
the list $L = (x_{1})$ and insert $x{i}$ in its correct place in the
list for $i = 2,3,\dots ,n$. Insertions can be done
\textbf{sequential} or preferentially by the \textbf{bisection}
method.\\
\textbf{GRAPHS}\\
A \textbf{graph $G$} consists of a finite set $V$, whose members are
called \textbf{vertices}, and a set $E$ of 2-subsets of $V$, whose
members are called \textbf{edges}. We usually write $G = (V,E)$ and
say that $V$ is the \textbf{vertex set} and $E$ is the \textbf{edge
  set}.\\
Let us say that two vertices $x$ and $y$ of a graph are $adjacent$
whenever $\{x,y\}$ is an edge. (We say also that $x$ and $y$ are
\textbf{neighbours}.) Then we can represent a graph $G = (V,E)$ by its
\textbf{adjacency list}, wherein each vertex $v$ heads a list of those
vertices adjacent to $v$.\\
Two graphs $G_{1}$ and $G_{2}$ are said to be \textbf{isomorphic} when
there is a bijection $\alpha$ from the vertex set of $G_{1}$ to the
vertex set of $G_{2}$ such that $\{\alpha(x), \alpha(y)\}$ is an edge
of $G_{2}$ if and only if $\{x, y\}$ is an edge of $G_{1}$. The
bijection $\alpha$ is said to be an \textbf{isomorphism}.\\
The \textbf{degree} of a vertex $v$ in a graph $G = (V,E)$ is the
number of edges of $G$ which contain $v$. We shall use the notation
$\delta(v)$ for the degree of $v$, so formally $\delta(v) = |D_{v}|$,
where $D_{v} = \{e \in E | v \in e \}$.\\
The sum of the values of the degree $\delta(v)$, taken over all the
vertices $v$ of a graph $G = (V,E)$, is equal to twice the number of
edges: $\displaystyle\sum_{v \in V} \delta(v) = 2|E|$.\\
\textit{The number of odd vertices is even}.\\
A \textbf{walk} in a graph $G$ is a sequence of vertices
$v_{1},v_{2},\dots , v_{k},$ such that $v_{i}$ and $v_{i+1}$ are
adjacent $(1 \leq i \leq k-1)$. If all its vertices are distinct, a
walk is called a \textbf{path}.\\
Suppose $G = (V,E)$ is a graph and the partition of $V$ corresponding
to the equivalence relation ~ is $V = V_{1} \cup V_{2} \cup \dots \cup
V_{r}$. Let $E_{i} (1 \leq i \leq r)$ denote the subset of $E$
comprising those edges whose ends are both in $V_{i}$. Then the graphs
$G_{i} = (V_{i}, E_{i})$ are called the \textbf{components} of $G$. If
$G$ has just one component, it is said to be \textbf{connected}.\\
We say that a graph $T$ is a \textbf{tree} if it has two properties:
(T1) $T$ is connected; (T2) there are no cycles in $T$.\\
If $T = (V,E)$ is a tree with at least two vertices, then: (T3) for
each pair of $x,y$ of vertices there is a unique path in $T$ from $x$
to $y$, (T4) the graph obtained from $T$ by removing any edge has two
components, each of which is a tree; (T5) $|E| = |V|-1$\\
A \textbf{vertex coloring} of a graph $G = (V,E)$ is a function $c: V
\to \mathbb{N}$ with the property that $c(x) /neq c(y)$ whenever
$\{x,y\} \in E$. The \textbf{chromatic number} of $G$, written
$\chi(G)$, is defined to be the least integer $k$ for which there is a
vertex coloring of $G$ using $k$ colors.\\
\textbf{Greedy algorithm} for vertex coloring. Not always perfect, but
ok. Suppose we have arranged the vertices in some order $v_{1}, v_{2},
\dots ,v_{n}$. We assign color 1 to $v_{i}$ for each $v_{i}(2 \leq i
\leq n)$ we form the set $S$ of colors assigned to vertices $v_{j}(1
\leq j < i)$ which is adjacent to $v_{i}$; and we give $v_{i}$ the
first color not in $S$.\\
If $G$ is a graph with maximum degree $k$, then (i) $\chi(G) \leq k +
1$, (ii) if $G$ is connected and not regular, $\chi(G) \leq k$.\\
A graph is bipartite if and only if it contains no cycles with odd length.\\
\textbf{TREES, SORTING, SEARCHING}\\
A vertex in a rooted tree is said to be a \textbf{leaf} if it is at
level $i (i \geq 0)$ and it is not adjacent to any vertices at level
$i + 1$. A vertex which is not a leaf is an \textbf{internal}
vertex. The \textbf{height} of a rooted tree is the maximum value of
$k$ for which level $k$ is not empty.\\
The height of an $m$-ary rooted tree with $l$ leaves is at least
$log_{m}l$.\\
The \textbf{heapsort} algorithm involves two stages. First, the
unsorted list is transformed into a special kind of list know as
\textit{heap}, and secondly, the heap is transformed into the sorted
list. The characteristic property of a heap is that each father is
smaller than his sons. Heapsort $O(n$ log($n))$.\\
Suppose that $G = (V, E)$ is a connected graph, and $T$ is a subset of
$E$ such that (i) every vertex of $G$ belongs to an edge in $T$; (ii)
the edges in $T$ form a tree. In this case, we say that $T$ is a
\textbf{spanning tree} for $G$.\\
\textbf{Minimum Spanning Tree MST} for the weighted graph $G$. A
simple algorithm for the MST problem is based on applying the greedy
principle to the tree-growing method given above. Specifically: at
each stage we add the \textit{cheapest} edge joining a new vertex to
the partial tree.\\
Let $G = (V,E)$ be a connected graph with weight function $w: E \to
\mathbb{N}$, and suppose that $T$ is a spanning tree for $G$
constructed by the greedy algorithm, then $w(T) \leq w((U)$ for any
spanning tree $U$ of $G$.\\
\textbf{Depth-First-Search}. Let $v$ be a vertex of the graph $G$ and
let $T$ be the subset of the edges of $G$ constructed according to the
DFS method. Then $T$ is a spanning tree for the component of $G$ which
contains $v$. (Stack procedure)\\
\textbf{Breadth-First-Search}. Let $v$ be a vertex of the graph $G$,
and let $T$ be  the subset of the edges of $G$ constructed according
to the BFS algorithm. Then $T$ is a spanning tree for the component of
$G$ which contains $v$. (Queue procedure)\\
\textbf{shortest path problem} Dijkstras algorithm, greedy\\
\textbf{BIPARTITE GRAPHS AND MATCHINGS}\\
Let $G = (X \cup Y, E)$ be a bipartite graph and let $\delta(v)$
denote the degree of a vertex $v$ in $G$. Then $\displaystyle\sum_{x
  \in X} \delta(x) = \displaystyle\sum_{y \in Y} \delta(y) = |E|$.\\
Let $G$ be a graph with edge set $E$. A coloring of $E$ is said to be
an $edge coloring$ of $G$ if any two edges containing the same vertex
have the different colors.\\
If $G = (X \cup Y, E)$ is a bipartite graph, then the minimum number
of colors needed for an edge coloring of $G$ is equal to the maximum
degree of $G$.\\
Any $m \times n$ latin rectangle with $1 \leq m \leq n$ can be
completed to form an $n \times n$ latin square.\\
Let $R$ be a partial $m \times p$ latin rectangle in which the symbols
$\{s_{1}, s_{2}, \dots,s{n}\}$ are used, and let $n_{R}(s_{i})$ denote
the number of times $s_{i}$ occours in $R(1 \leq i \leq n)$. Then $R$
can be completed to an $n \times n$ latin square if and only if
$n_{R}(s_{i}) \geq m + p - n$ $(1 \leq i \leq n)$.\\
A \textbf{matching} in a bipartite graph $G = (X \cup Y, E)$ is a
subset $M$ of $E$ with the property that no two edges in $M$ have a
common vertex.\\
We shall say that a matching $M$ is a \textbf{maximum matching} for $G
=(X \cup Y,E)$ if no other matching has a greated cardinality. If $|M|
= |X|$ (all the people get jobs), then we say that $M$ is a
\textbf{complete matching}.\\
If $G = (X \cup Y, E)$ and $A$ is a subset of $X$, let $J(A) = \{y \in
Y | xy \in E$ for some $x \in A\}$ so that $J(A)$ is the set of jobs
for which the people in $A$ are collectively qualified.\\
The bipartite graph $G = (X \cup Y, E)$ hasd a complete matching if an
oly if Hall's condition is satisfied, that is $|J(A)| \geq |A|$ for
all $A \subseteq X$.\\ 




% You can even have references
\rule{0.3\linewidth}{0.25pt}
\scriptsize
\bibliographystyle{abstract}
\bibliography{refFile}
\end{multicols}
\end{document}